\documentclass[a4paper, 12pt]{article}  % Change to report, book, etc. if needed

% Essential Packages
\usepackage[utf8]{inputenc}   % Handle UTF-8 encoding
\usepackage[T1]{fontenc}      % Proper font encoding
\usepackage{lmodern}          % Modern font
\usepackage{geometry}         % Page margins
\geometry{margin=1in}        % Set 1-inch margins

% Math and Symbols
\usepackage{amsmath, amssymb} % Math symbols and environments
\usepackage{physics}          % Common physics notation
\usepackage{siunitx}          % SI unit formatting

% Graphics and Figures
\usepackage{graphicx}         % Image inclusion
\usepackage{caption, subcaption} % Better figure captions
\usepackage{float}            % Control float positioning
\usepackage{listings}
\usepackage{color}

\lstset{language=Python}
\definecolor{dkgreen}{rgb}{0,0.6,0}
\definecolor{gray}{rgb}{0.5,0.5,0.5}
\definecolor{mauve}{rgb}{0.58,0,0.82}

\lstset{frame=tb,
  language=Python,
  aboveskip=3mm,
  belowskip=3mm,
  showstringspaces=false,
  columns=flexible,
  basicstyle={\small\ttfamily},
  numbers=none,
  numberstyle=\tiny\color{gray},
  keywordstyle=\color{blue},
  commentstyle=\color{dkgreen},
  stringstyle=\color{mauve},
  breaklines=true,
  breakatwhitespace=true,
  tabsize=3
}


% Tables
\usepackage{booktabs}         % Better table formatting
\usepackage{array}            % Extended table options



% Title and Author
\title{Spacecraft Mission}
\author{Alex Freeman}
\date{\today}

\begin{document}

\maketitle


This report will cover all parts of assignment one in the Aerospace 720 course

\section{Orbital Propagation}

\subsection{Solving Kepler's Equation}

\textbf{I)} We need to solve Kepler's Equation using numerical methods. Using the Newton-Rasphon Method we can take the eccentricity and mean anamoly as 
inputs and numerically solve for the eccentric anamoly. 

\begin{lstlisting}
    def Kepler(e, M, tol = 1e-12, max_i = 1000):
    E = M                               # Guess solution
    for i in range(max_i):
        f_E = E - e * np.sin(E) - M     # Define the function in terms of f(E) = 0
         f_prime = 1 - e * np.cos(E)    # Derive the function in terms of E
        del_E = f_E / f_prime           
        E_new = E - del_E               # Calculate the new eccentric anamoly
        if np.abs(del_E) < tol:         # If the value is within the set tolerance 
            theta = 2*np.arctan(np.tan(E_new/2) * ((1+e)/(1-e))**(0.5))
            return theta                # Return true anamoly
        E = E_new
\end{lstlisting}

\noindent \textbf{II)} If we set the tolerance to $1e-12$, we can compute the true anamoly of the asteroid 
at $t_0$ and $t_0 + 100$ days. A\_ae0 is the OBJ data of the asteroid, it is an array.
\begin{lstlisting}
trueAnamoly_asteroidt_0 = Kepler(A_ae0[2], A_ae0[6])  
meanAnamolyt_100 = get_mean_anamoly(100*(3600*24), A_ae0[6], A_ae0[1])
trueAnamoly_asteroidt_100 = Kepler(A_ae0[2], meanAnamolyt_100)
\end{lstlisting}
Printing these values gives the following



\section{Mathematics}
To derive the necessary state function we have:
\begin{equation}
    \ddot{\mathbf{r}} = -\frac{\mu}{r^3} \mathbf{r}
\end{equation}
From here we know that $\frac{dv}{dt} = \dot{r}$ giving:
\begin{equation}
    \frac{d\mathbf{v}}{dt} = -\frac{\mu}{r^3} \mathbf{r}
\end{equation}
Expanding each vector as three-dimensional components in $x,y,z$:

\begin{equation}
    \frac{dx}{dt} = v_{x},  \frac{dy}{dt} = v_{y},   \frac{dz}{dt} = v_{z}
\end{equation}


\begin{equation}
    \frac{dv_{x}}{dt} = -\frac{\mu}{r^3} x, \frac{dv_{y}}{dt} = -\frac{\mu}{r^3} y, \frac{dv_{z}}{dt} = -\frac{\mu}{r^3} z
\end{equation}
Where $r = \sqrt{x^{2} + y^{2} + z^{2}}$
\newline
We can now define a state vector $\mathbf{\bar{X}}$


\begin{equation}
    \mathbf{\bar{X}} = 
    \begin{bmatrix}
        x \\
        y \\
        z \\
        v_x \\
        v_y \\
        v_z
        \end{bmatrix}
\end{equation}

Finally, deriving this state vector gives the following:

\begin{equation}
    \mathbf{\dot{\bar{X}}} = 
    \begin{bmatrix}
        v_x \\
        v_y \\
        v_z \\
        -\frac{\mu}{r^3} x \\
        -\frac{\mu}{r^3} y \\
        -\frac{\mu}{r^3} z
        \end{bmatrix}
\end{equation}



\section{Figures}
Figure example:
\begin{figure}[H]
    \centering
    \includegraphics[width=0.6\textwidth]{example-image}
    \caption{Example figure caption.}
    \label{fig:example}
\end{figure}

\section{Tables}
\begin{table}[H]
    \centering
    \begin{tabular}{l c r}
        \toprule
        Left & Center & Right \\
        \midrule
        A & B & C \\
        1 & 2 & 3 \\
        \bottomrule
    \end{tabular}
    \caption{Example table.}
    \label{tab:example}
\end{table}

\end{document}
