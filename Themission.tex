\documentclass[a4paper, 12pt]{article}  % Change to report, book, etc. if needed

% Essential Packages
\usepackage[utf8]{inputenc}   % Handle UTF-8 encoding
\usepackage[T1]{fontenc}      % Proper font encoding
\usepackage{lmodern}          % Modern font
\usepackage{geometry}         % Page margins
\geometry{margin=1in}        % Set 1-inch margins

% Math and Symbols
\usepackage{amsmath, amssymb} % Math symbols and environments
\usepackage{physics}          % Common physics notation
\usepackage{siunitx}          % SI unit formatting

% Graphics and Figures
\usepackage{graphicx}         % Image inclusion
\usepackage{caption, subcaption} % Better figure captions
\usepackage{float}            % Control float positioning
\usepackage{listings}
\usepackage{color}

\lstset{language=Python}
\definecolor{dkgreen}{rgb}{0,0.6,0}
\definecolor{gray}{rgb}{0.5,0.5,0.5}
\definecolor{mauve}{rgb}{0.58,0,0.82}

\lstset{frame=top,
  language=Python,
  aboveskip=3mm,
  belowskip=3mm,
  showstringspaces=false,
  columns=flexible,
  basicstyle={\small\ttfamily},
  numbers=none,
  numberstyle=\tiny\color{gray},
  keywordstyle=\color{blue},
  commentstyle=\color{dkgreen},
  stringstyle=\color{mauve},
  breaklines=true,
  breakatwhitespace=true,
  tabsize=3
}


% Tables
\usepackage{booktabs}         % Better table formatting
\usepackage{array}            % Extended table options



% Title and Author
\title{Spacecraft Mission}
\author{Alex Freeman}
\date{\today}

\begin{document}

\maketitle


This report will cover all parts of assignment one in the Aerospace 720 course

\section{Orbital Propagation}

\subsection{Solving Kepler's Equation}

\textbf{I)} We need to solve Kepler's Equation using numerical methods. Using the Newton-Rasphon Method we can take the eccentricity and mean anamoly as 
inputs and numerically solve for the eccentric anamoly. 

\begin{lstlisting}
def Kepler(e, M, tol = 1e-12, max_i = 1000):
    E = M                               # Guess solution
    for i in range(max_i):
        f_E = E - e * np.sin(E) - M     # Define the function in terms of f(E) = 0
         f_prime = 1 - e * np.cos(E)    # Derive the function in terms of E
        del_E = f_E / f_prime           
        E_new = E - del_E               # Calculate the new eccentric anamoly
        if np.abs(del_E) < tol:         # If the value is within the set tolerance 
            theta = 2*np.arctan(np.tan(E_new/2) * ((1+e)/(1-e))**(0.5))
            return theta                # Return true anamoly
        E = E_new
\end{lstlisting}
\vspace{0.5cm}
\noindent \textbf{II)} If we set the tolerance to $1e-12$, we can compute the true anamoly of the asteroid 
at $t_0$ and $t_0 + 100$ days. A\_ae0 is the OBJ data of the asteroid, it is an array.
\begin{lstlisting}
trueAnamoly_asteroidt_0 = Kepler(A_ae0[2], A_ae0[6])  
meanAnamolyt_100 = get_mean_anamoly(100*(3600*24), A_ae0[6], A_ae0[1])
trueAnamoly_asteroidt_100 = Kepler(A_ae0[2], meanAnamolyt_100)
\end{lstlisting}
Printing these values gives that the true anamoly $\theta_{t_0} = 1.4246$ and $\theta_{t_0 + 100} = 2.1369$. 
Where these answers are in radians.

\vspace{0.5cm}
\noindent \textbf{III)} Now I have created a function that takes in a state of orbital elements and returns 
the position and velocity vectors at that point. It uses a rotation matrix to convert from 
the perifocal frame to the ECI frame. This is defined via $i, \omega$, and $\Omega$ terms and is 
calculated using the defind matricies in the appendix.

\begin{lstlisting}   
def COE2RV(arr, mu):
    a, e, i, Omega, omega, theta_var = arr[0:6]
    h = np.sqrt(mu * a * (1 - e**2))
    r = a*(1-(e**2))/(1 + e*np.cos(theta_var))

    arr_r = np.array([r*np.cos(theta_var), r*np.sin(theta_var), 0])
    arr_v = (mu/h)* np.array([-np.sin(theta_var), e + np.cos(theta_var), 0])

    # Rotate position and velocity from perifocal to inertial frame using the 
    # transfomration matrix
    R_matrix = rotation_matrix(i, Omega, omega)

    r_ijk = R_matrix @ arr_r
    v_ijk = R_matrix @ arr_v
    return r_ijk, v_ijk
\end{lstlisting}
Using this code we can output the state vector at some time $t$. The first three values 
are the $x,y,z$ positions in \textbf{km}. The last three are the velocity values in the $x,y,z$ direction 
in \textbf{km/s}.
\[
\begin{tabular}{c@{\hspace{2cm}}c}
\textbf{At } $t_0$: &
\textbf{At } $t_0 + 100$: \\
$\mathbf{\bar{X}} =
\begin{bmatrix}
x = -1.1694365\mathrm{e}{+08} \\
y =  1.53462780\mathrm{e}{+08} \\
z = -6.7446087\mathrm{e}{+06} \\
v_x = -3.1710203\mathrm{e}{+01} \\
v_y = -3.6285380\mathrm{e}{+00} \\
v_z = -1.8931546\mathrm{e}{+00}
\end{bmatrix}$ &
$\mathbf{\bar{X}} =
\begin{bmatrix}
x = -3.2057997\mathrm{e}{+08} \\
y =  6.72659396\mathrm{e}{+07} \\
z = -1.8991445\mathrm{e}{+07} \\
v_x = -1.6964807\mathrm{e}{+01} \\
v_y = -1.2943780\mathrm{e}{+01} \\
v_z = -1.0284663\mathrm{e}{+00}
\end{bmatrix}$
\end{tabular}
\]
\textbf{IV)} Next, I have written a function called "Ephemeris". It returns the position and velocity at some time t.
\begin{lstlisting}
def Ephemeris(t, OBJdata, mu):
    time, a, e, i, Omega, omega, mean_anamoly = OBJdata[0:7]
    nu_t = (mu / (a**3))**0.5
    t = t - t_0_days*days_convert
    mean_anamoly_t = mean_anamoly + nu_t * (t)

    h = np.sqrt(mu * a * (1 - e**2))
    theta_var = Kepler(e, mean_anamoly_t)
    r = a*(1-(e**2))/(1 + e*np.cos(theta_var))
    arr_r = np.array([r*np.cos(theta_var), r*np.sin(theta_var), 0])
    arr_v = (mu/h)* np.array([-np.sin(theta_var), e + np.cos(theta_var), 0])
    R_matrix = rotation_matrix(i, Omega, omega)
    r_ijk = R_matrix @ arr_r
    v_ijk = R_matrix @ arr_v
    
    return r_ijk, v_ijk
\end{lstlisting}

\noindent Using this code we can calculate the position and velocity vectors in the Sun's frame of reference. 

\begin{figure}[H]
    \centering
    \begin{minipage}{0.48\textwidth}
        \centering
        \includegraphics[width=\textwidth]{114-pos.png}
        \caption{Position vectors of Earth and the Asteroid's for a full Earth orbital period}
        \label{fig:pos}
    \end{minipage}
    \hfill
    \begin{minipage}{0.48\textwidth}
        \centering
        \includegraphics[width=\textwidth]{114-vel.png}
        \caption{Velocity vectors of Earth and the Asteroid's for a full Earth orbital period}
        \label{fig:vel}
    \end{minipage}
\end{figure}

\noindent Next we can plot the seperation of the two bodies over ten years. Doing this we get the following graph

\begin{figure}[H]
    \centering
    \includegraphics[width=0.6\textwidth]{115-dis.png}
    \caption{Distance between Earth and 2024 YR4 asteroid in kilometers}
    \label{fig:example}
\end{figure}

\noindent We can see how the distance is sinusoidal in nature and repeats in an oscillatory fashion. 
This graph shows three distinct peaks, where the middle one is the greatest. We can examine the sydonic period of 
the two bides by examing the equation of the two periods $\frac{1}{T_{syd}} = |\frac{1}{T_{Earth}} - \frac{1}{T_{Asteroid}}|$. And using the orbital periods,
we can see that roughly every 1.33 years, the two bodies are at their closest approach. This lines up well with the graph produced,
showing the magnitude in their seperation. 


\subsection{Numerical Integration}
To derive the necessary state function we have:
\begin{equation}
    \ddot{\mathbf{r}} = -\frac{\mu}{r^3} \mathbf{r}
\end{equation}
From here we know that $\frac{dv}{dt} = \dot{r}$ giving:
\begin{equation}
    \frac{d\mathbf{v}}{dt} = -\frac{\mu}{r^3} \mathbf{r}
\end{equation}
Expanding each vector as three-dimensional components in $x,y,z$:

\begin{equation}
    \frac{dx}{dt} = v_{x},  \frac{dy}{dt} = v_{y},   \frac{dz}{dt} = v_{z}
\end{equation}


\begin{equation}
    \frac{dv_{x}}{dt} = -\frac{\mu}{r^3} x, \frac{dv_{y}}{dt} = -\frac{\mu}{r^3} y, \frac{dv_{z}}{dt} = -\frac{\mu}{r^3} z
\end{equation}
Where $r = \sqrt{x^{2} + y^{2} + z^{2}}$
\newline
We can now define a state vector $\mathbf{\bar{X}}$


\begin{equation}
    \mathbf{\bar{X}} = 
    \begin{bmatrix}
        x \\
        y \\
        z \\
        v_x \\
        v_y \\
        v_z
        \end{bmatrix}
\end{equation}

Finally, deriving this state vector gives the following:

\begin{equation}
    \mathbf{\dot{\bar{X}}} = 
    \begin{bmatrix}
        v_x \\
        v_y \\
        v_z \\
        -\frac{\mu}{r^3} x \\
        -\frac{\mu}{r^3} y \\
        -\frac{\mu}{r^3} z
        \end{bmatrix}
\end{equation}
Now we can use a function to define the right-hand side of this equation
\begin{lstlisting}
def TBP_ECI(t, state_X, mu):
    x, y, z, vx, vy, vz = state_X  # Unpack state vector
    r = np.sqrt(x**2 + y**2 + z**2)  # Compute radius
    ax, ay, az = -mu * x / r**3, -mu * y / r**3, -mu * z / r**3  # Acceleration components
    return [vx, vy, vz, ax, ay, az]  # Return derivatives
\end{lstlisting}
With this function we can use $SciPy's$ integration feature with solve\_ivp. We can pass 
through a set of initial conditions: a position and velocity vector. It passes through the 
gravitational parameter, $\mu$ as an argument when solving the differential system. Furthermore, 
it uses the Runge-Kutta 45 method to integrate. 
\begin{lstlisting}
r0 = np.linalg.norm(X0[:3])  # Initial distance from Earth's center (km)
v0 = np.linalg.norm(X0[3:])  # Initial speed (km/s)
a = 1/(2/r0 - v0**2/mu_earth)  # Semi-major axis (km)
T = 2 * np.pi * np.sqrt(a**3/mu_earth)  # Orbital period (s)

# Set integration time span for two orbital periods
t_start = 0
t_end = 2 * T  # Two orbital periods
time_step = 10  # Output every 10 seconds
t_eval = np.arange(t_start, t_end, time_step)  

# Solve the system using solve_ivp with strict tolerances
solution = solve_ivp(
    TBP_ECI, (t_start, t_end), X0, t_eval=t_eval, method='RK45',
    args=(mu_earth,), rtol=1e-12, atol=1e-12
)

# Extract components
x, y, z = solution.y[0], solution.y[1], solution.y[2]
vx, vy, vz = solution.y[3], solution.y[4], solution.y[5]
\end{lstlisting}


\begin{figure}[H]
    \centering
    \includegraphics[width=0.6\textwidth]{123-3d.png}
    \caption{Orbital trajectory in ECI frame with initial conditions given from $X_0$}
    \label{fig:example}
\end{figure}

\begin{figure}[H]
    \centering
    \includegraphics[width=0.6\textwidth]{124-energies.png}
    \caption{How Kinetic, Potential, and Total energies change throughout the orbit }
    \label{fig:example}
\end{figure}

This graph shows the sinusoidal nature of the different specific energies. Importantly, we can see how the 
kinetic and potential energies always sum to the same value. This ensures that the total energy remains constant, which we would expect 
as this is a closed system with no external forces. 

\begin{figure}[H]
    \centering
    \includegraphics[width=0.6\textwidth]{124-L.png}
    \caption{Graph shows the change in the angular momentum over the orbit is zero}
    \label{fig:example}
\end{figure}
\noindent Here we see this graph depicting how the angular momentum changes over time. It shows how the angular momentum stays 
at a constant value and remains constant. This makes sense because the angular momentum is defined as the cross product
between the position and velocity vector.
\setcounter{equation}{0}
\begin{equation}
\frac{dh}{dt} = \frac{d}{dt}(\bar{r} \times \dot{\bar{r}}) \implies \frac{dh}{dt} = \bar{r} \times \frac{d\dot{\bar{r}}}{dt}
\end{equation}

\begin{equation}
\frac{dh}{dt} = \bar{r} \times -\frac{\mu}{r^{3}}\bar{r} \implies \frac{dh}{dt} = 0
\end{equation}
\vspace{0.2cm}

So yes, these diagrams make physical sense, based upon the conservation laws of both Energy and Angular Momentum


\section{Orbital maneuvers and mission design}
\subsection{Reachability Analysis}

\begin{lstlisting}
def RV2COE(state_x, mu):
    r_vec = state_x[0:3]
    v_vec = state_x[3:6]
    r_mag = norm(r_vec)
    v_mag = norm(v_vec)
    
    a = r_mag / (2 - (r_mag*v_mag**2/mu))
    h = np.cross(r_vec, v_vec)
    h_mag = norm(h)
    e_vec = np.cross(v_vec, h) / mu - r_vec / r_mag
    e_mag = norm(e_vec)
    i = np.arccos(h[2]/h_mag)
    
    n_vec = np.cross(k_hat, h)
    n_mag = norm(n_vec)
    n_hat = n_vec / n_mag
    Omega_raan = np.arccos(n_hat[0])
    if n_hat[1] < 0:
        Omega_raan = 2*np.pi - Omega_raan
    
    omega = np.arccos(np.dot(n_vec, e_vec)/(n_mag * e_mag))
    if e_vec[2] < 0:
        omega = 2*np.pi - omega
        
    cos_theta = np.dot(r_vec, e_vec) / (r_mag * e_mag)
    cos_theta = np.clip(cos_theta, -1.0, 1.0)  
    theta = np.arccos(cos_theta)
    if np.dot(r_vec, v_vec) < 0:
        theta = 2*np.pi - theta
    
    return np.array([a, e_mag, i, Omega_raan, omega, theta])
\end{lstlisting}

Using this function, we can report the COE state for the initial condition. Inputting the vector $\bar{X}$, returns the following elements.

\begin{center}
    \textbf{At } $X_0$:
    COE =
    $\begin{bmatrix}
    a = 7.17813700\mathrm{e}{+03} \\
    e =  7.00000000\mathrm{e}{-02} \\
    i = 1.67551608\mathrm{e}{+00} \\
    \Omega = 3.49065850\mathrm{e}{-01} \\
    \omega = 7.85398163\mathrm{e}{-01} \\
    \theta = 6.28318529\mathrm{e}{+00}
    \end{bmatrix}$  
\end{center}

\begin{lstlisting}
def rotate_matrix(state_x):
    r_vec = state_x[0:3]
    v_vec = state_x[3:6]
    
    r_hat = r_vec/(norm(r_vec))
    

    h_vec = np.cross(r_vec, v_vec)
    h_hat = h_vec / np.linalg.norm(h_vec)  
    t_hat = np.cross(h_hat, r_hat)  

    rotation = np.column_stack((r_hat, t_hat, h_hat))
    
    return rotation
\end{lstlisting}

With this rotation matrix now defined at every point along the orbit, I can define a function named "impulse". This function
takes a rotation matrix, direciton and an initial state as input parameters. It calculates the impulse in the direction of either, 
radius, transverse or normal (which are calculated via 3-D basis vectors). Then the rotation matrix is applied to the impulse to convert
it from the RTN frame to the ECI frame. It adds this vector to the velocity components of the state and then calculates the orbital 
elements of this final state and the inputted initial state to find the difference between components. 
\begin{lstlisting}
def impulse(r_matrix, direct, initial_state):
    dv_ = np.dot(delta_v, direct)

    impulse_eci =  r_matrix @ dv_

    state_final = initial_state.copy()
    state_final[3:] += impulse_eci
    oElements_initial = RV2COE(initial_state, mu_earth)
    oElements_final = RV2COE(state_final, mu_earth)

    coe_diff = oElements_final - oElements_initial

    return coe_diff, oElements_final
\end{lstlisting}

Using these combinations of functions we can apply an impulse in the three different directions and calculate the changes.
\begin{table}[h]
    \centering
    \small
    \setlength{\tabcolsep}{4pt}
    \renewcommand{\arraystretch}{1.1}
    \begin{tabular}{lrrrrrr}
    \hline
    Impulse Type & $\Delta a$ & $\Delta e$ & $\Delta i$ & $\Delta \Omega$ & $\Delta \omega$ & $\Delta M$ \\
    \hline
    Radial     & 0.012927 & $1.28\mathrm{e}{-5}$ & 0 & 0 & -0.0191214 & -6.26406 \\
    Transverse & 20.7374  & 0.00267900   & 0 & 0 & 0 & 0.0000000149012 \\
    Normal     & 0.012927 & $1.67\mathrm{e}{-6}$ & $8.85\mathrm{e}{-4}$ & 0.000889606 & 0.0000933805 & 0.0000000149012 \\
    \hline
    \end{tabular}
    \caption{Changes in orbital elements after applying impulses in radial, transverse, and normal directions.}
    \label{tab:orbital_changes}
    \end{table}
    

    \begin{figure}[h]
        \centering
    
        % First row
        \begin{subfigure}[t]{0.45\textwidth}
            \centering
            \includegraphics[width=\linewidth]{del_a.png}
            \caption{Graph 1}
            \label{fig:graph1}
        \end{subfigure}
        \hspace{0.05\textwidth}
        \begin{subfigure}[t]{0.45\textwidth}
            \centering
            \includegraphics[width=\linewidth]{del_e.png}
            \caption{Graph 2}
            \label{fig:graph2}
        \end{subfigure}
    
        \vspace{1em}
    
        % Second row
        \begin{subfigure}[t]{0.45\textwidth}
            \centering
            \includegraphics[width=\linewidth]{del_i.png}
            \caption{Graph 3}
            \label{fig:graph3}
        \end{subfigure}
        \hspace{0.05\textwidth}
        \begin{subfigure}[t]{0.45\textwidth}
            \centering
            \includegraphics[width=\linewidth]{del_Omega.png}
            \caption{Graph 4}
            \label{fig:graph4}
        \end{subfigure}
    
        \vspace{1em}
    
        % Third row
        \begin{subfigure}[t]{0.45\textwidth}
            \centering
            \includegraphics[width=\linewidth]{del_lomega.png}
            \caption{Graph 5}
            \label{fig:graph5}
        \end{subfigure}
        \hspace{0.05\textwidth}
        \begin{subfigure}[t]{0.45\textwidth}
            \centering
            \includegraphics[width=\linewidth]{del_theta.png}
            \caption{Graph 6}
            \label{fig:graph6}
        \end{subfigure}
    
        \caption{Six graphs arranged in three rows, two per row.}
        \label{fig:six_graphs}
    \end{figure}


\section{Tables}
\begin{table}[H]
    \centering
    \begin{tabular}{l c r}
        \toprule
        Left & Center & Right \\
        \midrule
        A & B & C \\
        1 & 2 & 3 \\
        \bottomrule
    \end{tabular}
    \caption{Example table.}
    \label{tab:example}
\end{table}

\section{Appendix}
\begin{lstlisting}
    # -*- coding: utf-8 -*-
"""
Created on Tue Mar 25 22:08:35 2025

@author: alexa
"""



#Import Python files
import matplotlib.pyplot as plt
import numpy as np
from mpl_toolkits.mplot3d import Axes3D
from scipy.integrate import solve_ivp
from scipy import signal

show_plots = True



#Assignment Data

mu_sun = 1.3271244*10**11
mu_earth = 3.986*10**5
radius_earth = 6378.14
J_2_earth = 1082.63*10**-6
mu_moon = 4902.8
radius_moon = 1737.4
r_mean = 384400


'''
Functions
'''

def norm(vec):
    return np.linalg.norm(vec)

pi = np.pi
def radians(deg):
    rads = deg * (pi/180)
    return rads


def get_mean_anamoly(del_t, M_old, a):
    nu_t = (mu_sun / (a**3))**0.5
    mean_anamoly_t = M_old + nu_t * (del_t)
    
    return mean_anamoly_t

def time_orbit(a, μ):
    T = 2 * np.pi * np.sqrt(a**3 / μ)
    time = T 
    return time

def rotation_matrix(i, Omega, omega):
    cos_O, sin_O = np.cos(Omega), np.sin(Omega)
    cos_i, sin_i = np.cos(i), np.sin(i)
    cos_w, sin_w = np.cos(omega), np.sin(omega)
    
    R = np.array([
        [cos_O * cos_w - sin_O * sin_w * cos_i, -cos_O * sin_w - sin_O * cos_w * cos_i, sin_O * sin_i],
        [sin_O * cos_w + cos_O * sin_w * cos_i, -sin_O * sin_w + cos_O * cos_w * cos_i, -cos_O * sin_i],
        [sin_w * sin_i, cos_w * sin_i, cos_i]
    ])
    
    return R

'''
Newton-Rasphon Method to return the eccentric anamoly to a certain tolerance
'''

#1.1.1 ----------------------------------------------------------------------


def Kepler(e, M, tol = 1e-12, max_i = 1000):
    E = M
    
    
    for i in range(max_i):
        
        f_E = E - e * np.sin(E) - M
        f_prime = 1 - e * np.cos(E)
        
        del_E = f_E / f_prime
        
        
        E_new = E - del_E
        if np.abs(del_E) < tol:
            theta = 2*np.arctan(np.tan(E_new/2) * ((1+e)/(1-e))**(0.5))
            return theta
        E = E_new
 
        



    
E_ae0 = [2460705.5, 1.495988209443421E+08, 1.669829008180246E-02, radians(3.248050135173038E-03), 
          radians(1.744712892867145E+02), radians(2.884490093009512E+02), radians(2.621190445180298E+01)]


A_ae0 = [2460705.5, 3.764419202360106E+08, 6.616071771587672E-01, radians(3.408286057191753),
          radians(2.713674649188756E+02), radians(1.343644678984687E+02), radians(1.693237490356061E+01)]



#1.1.2 ----------------------------------------------------------------------


trueAnamoly_asteroidt_0 = Kepler(A_ae0[2], A_ae0[6])
meanAnamolyt_100 = get_mean_anamoly(100*(3600*24), A_ae0[6], A_ae0[1])
trueAnamoly_asteroidt_100 = Kepler(A_ae0[2], meanAnamolyt_100)

# print(trueAnamoly_asteroidt_0)
# print(trueAnamoly_asteroidt_100)



Obj2_t0 = A_ae0.copy()
Obj2_t0[6] = trueAnamoly_asteroidt_0
Obj2_t0 = Obj2_t0[1:]


Obj2_t100 = A_ae0.copy()
Obj2_t100[6] = trueAnamoly_asteroidt_100
Obj2_t100 = Obj2_t100[1:]

#1.1.3 ----------------------------------------------------------------------

t_0 = 2460705.5*(3600*24)

def COE2RV(arr, mu):
    a, e, i, Omega, omega, theta_var = arr[0:6]
    h = np.sqrt(mu * a * (1 - e**2))
    r = a*(1-(e**2))/(1 + e*np.cos(theta_var))
    
    arr_r = np.array([r*np.cos(theta_var), r*np.sin(theta_var), 0])
    arr_v = (mu/h)* np.array([-np.sin(theta_var), e + np.cos(theta_var), 0])

    #Rotate position and velocity from perifocal to inertial frame using the transfomration matrix
    R_matrix = rotation_matrix(i, Omega, omega)
   
    r_ijk = R_matrix @ arr_r
    v_ijk = R_matrix @ arr_v
    return r_ijk, v_ijk




state_vector_0 = np.array(COE2RV(Obj2_t0, mu_sun))
print(state_vector_0) # t = t0



state_vector_100 = np.array(COE2RV(Obj2_t100, mu_sun))
print(state_vector_100) # t = t0 + 100

# print('\n')
# print(state_vector_0 - state_vector_100)


days_convert = 3600*24
#1.1.4 ----------------------------------------------------------------------

def Ephemeris(t, OBJdata, mu):

    time, a, e, i, Omega, omega, mean_anamoly = OBJdata[0:7]
    nu_t = (mu / (a**3))**0.5
    t = t - t_0_days*days_convert
    mean_anamoly_t = mean_anamoly + nu_t * (t)

    h = np.sqrt(mu * a * (1 - e**2))
    theta_var = Kepler(e, mean_anamoly_t)
    r = a*(1-(e**2))/(1 + e*np.cos(theta_var))
    
    arr_r = np.array([r*np.cos(theta_var), r*np.sin(theta_var), 0])
    arr_v = (mu/h)* np.array([-np.sin(theta_var), e + np.cos(theta_var), 0])
    R_matrix = rotation_matrix(i, Omega, omega)
    r_ijk = R_matrix @ arr_r
    v_ijk = R_matrix @ arr_v
    
    return r_ijk, v_ijk


years_shown_i = 1

t_0_days = 2460705

t_array = days_convert*np.arange(0,years_shown_i*365, 1)


x_earth = np.zeros((6,len(t_array)))
x_asteroid = np.zeros((6,len(t_array)))


    
for r in range(len(t_array)):
    x_earth[0:6, r] = np.hstack(Ephemeris(t_array[r], E_ae0, mu_sun))
    x_asteroid[0:6, r] = np.hstack(Ephemeris(t_array[r], A_ae0, mu_sun))
    


time_Earth = time_orbit(E_ae0[1], mu_sun)
orbital_percent_E = (t_array / time_Earth) 


plt.plot(orbital_percent_E,[norm(x_earth[0:3, t]) for t in range(len(t_array))], label="|r| - Earth", color='b')
plt.plot(orbital_percent_E, [norm(x_asteroid[0:3, t]) for t in range(len(t_array))], label="|r| - Asteroid ", color='r')
plt.legend()


if show_plots:
    plt.show()
else:
    plt.close()

#1.1.5 ----------------------------------------------------------------------


years_shown = 10
t_total = days_convert*np.arange(0,years_shown*365, 1)



normed_diff = []
for t in t_total:
    
    normed_diff.append(norm(Ephemeris(t,E_ae0, mu_sun)[0] - Ephemeris(t,A_ae0, mu_sun)[0]))
    
    
plt.figure()
plt.plot(t_total*(10/t_total[-1]), normed_diff)
plt.title("Absolute distance seperation of Earth and 2024 YR4")
plt.xlabel("Time (years)")
plt.ylabel("Distance Seperation (km)")
if show_plots:
    plt.show()
else:
    plt.close()
    



#1.2.1 ----------------------------------------------------------------------

x0, y0, z0 = 4604.49276873138, 1150.81472538679, 4694.55079634563   # km
vx0, vy0, vz0 = -5.10903235110107 , -2.48824074138143 ,5.62098648967432   # km/s

# Pack initial state vector
X0 = [x0, y0, z0, vx0, vy0, vz0]

#1.2.2 ----------------------------------------------------------------------

def TBP_ECI(t, state_X, mu):
    x, y, z, vx, vy, vz = state_X  # Unpack state vector
    r = np.sqrt(x**2 + y**2 + z**2)  # Compute radius
    ax, ay, az = -mu * x / r**3, -mu * y / r**3, -mu * z / r**3  # Acceleration components
    return [vx, vy, vz, ax, ay, az]  # Return derivatives



#1.2.3 ----------------------------------------------------------------------

r0 = np.linalg.norm(X0[:3])  # Initial distance from Earth's center (km)
v0 = np.linalg.norm(X0[3:])  # Initial speed (km/s)
a = 1 / (2 / r0 - v0**2 / mu_earth)  # Semi-major axis (km)
T = 2 * np.pi * np.sqrt(a**3 / mu_earth)  # Orbital period (s)

# Set integration time span for two orbital periods
t_start = 0
t_end = 2 * T  # Two orbital periods
time_step = 10  # Output every 10 seconds
t_eval = np.arange(t_start, t_end, time_step)  

# Solve the system using solve_ivp with strict tolerances
solution = solve_ivp(
    TBP_ECI, (t_start, t_end), X0, t_eval=t_eval, method='RK45',
    args=(mu_earth,), rtol=1e-12, atol=1e-12
)

# Extract components
x, y, z = solution.y[0], solution.y[1], solution.y[2]
vx, vy, vz = solution.y[3], solution.y[4], solution.y[5]

r = np.sqrt(x**2 + y**2 + z**2)

# Compute speed


v_lin = np.sqrt(vx**2 + vy**2 + vz**2)



# 3D Trajectory Plot
fig = plt.figure(figsize=(8, 8))
ax = fig.add_subplot(111, projection='3d')

#Plot Earth as a sphere
earth_radius = 6378  # km (mean Earth radius)
u, v = np.mgrid[0:2*np.pi:50j, 0:np.pi:25j]
X_earth = earth_radius * np.cos(u) * np.sin(v)
Y_earth = earth_radius * np.sin(u) * np.sin(v)
Z_earth = earth_radius * np.cos(v)
ax.plot_surface(X_earth, Y_earth, Z_earth, color='b', alpha=0.3)

# Plot the orbit
ax.plot(x, y, z, label="Orbit Path", color='r')

# Labels and title
ax.set_xlabel("X (km)")
ax.set_ylabel("Y (km)")
ax.set_zlabel("Z (km)")
ax.set_title("Orbital Trajectory in ECI Frame")
ax.legend()

# Set limits to give a clear view
ax.set_xlim([-2*r0, 2*r0])
ax.set_ylim([-2*r0, 2*r0])
ax.set_zlim([-2*r0, 2*r0])

if show_plots:
    plt.show()
else:
    plt.close()    
    
    
#1.2.4 ----------------------------------------------------------------------


plt.figure()

KE = 0.5 * v_lin**2
PE = -mu_earth / r
E_total = KE + PE


x, y, z = solution.y[0], solution.y[1], solution.y[2]
vx, vy, vz = solution.y[3], solution.y[4], solution.y[5]
# Compute Specific Angular Momentum (km²/s)
h_array = np.sqrt(((y * vz) - (z * vy))**2 + ((z * vx) - (x * vz))**2 + ((x * vy) - (y * vx))**2)

# Convert time to hours for better readability
time_hours = solution.t / 3600

# Plot Specific Energies
plt.figure(figsize=(10, 5))

plt.subplot(1, 2, 1)
plt.plot(time_hours, KE, label="Kinetic Energy", color='r')
plt.plot(time_hours, PE, label="Potential Energy", color='b')
plt.plot(time_hours, E_total, label="Total Energy", color='k', linestyle='dashed')
plt.xlabel("Time (hours)")
plt.ylabel("Specific Energy (km²/s²)")
plt.title("Specific Energies Over Time")
plt.legend()
plt.grid()

if show_plots:
    plt.show()
else:
    plt.close()

# Plot Specific Angular Momentum
plt.figure()
plt.plot(time_hours, h_array.round(2),'r', label="Specific Angular Momentum")
plt.xlabel("Time (hours)")
plt.ylabel("Angular Momentum (km²/s)")
plt.title("Specific Angular Momentum Over Time")
plt.legend()
plt.grid()

if show_plots:
    plt.show()
else:
    plt.close()
    
    
#2.1.1 ----------------------------------------------------------------------

k_hat = np.array([0,0,1])


def RV2COE(state_x, mu):
    #x,y,z,vx,vy,vz = state_x
    r_vec = state_x[0:3]
    v_vec = state_x[3:6]
    r_mag = norm(r_vec)
    v_mag = norm(v_vec)
    
    a = r_mag / (2 - (r_mag*v_mag**2/mu))
    
    h = np.cross(r_vec, v_vec)
    h_mag = norm(h)
    e_vec = np.cross(v_vec, h) / mu - r_vec / r_mag
    e_mag= np.linalg.norm(e_vec)
    
    i = np.arccos(h[2]/h_mag)
    
    n_vec = np.cross(k_hat, h)
    n_mag = norm(n_vec)
    n_hat = n_vec / n_mag
    Omega_raan = np.arccos(n_hat[0])
    
    if n_hat[1] < 0:
        Omega_raan = 2*np.pi - Omega_raan
    
    
    omega = np.arccos(np.dot(n_vec, e_vec)/(n_mag * e_mag))
    
    if e_vec[2] < 0:
        omega = 2*np.pi - omega
        
    
    cos_theta = np.dot(r_vec, e_vec) / (r_mag * e_mag)
    cos_theta = np.clip(cos_theta, -1.0, 1.0)  # ensures it's in valid domain
    theta = np.arccos(cos_theta)
    
    if np.dot(r_vec, v_vec) < 0:
        theta = 2*np.pi - theta
    
    
    return np.array([a, e_mag, i, Omega_raan, omega, theta])
    
    


#print(RV2COE(X0, mu_earth))
    
#2.1.2 ----------------------------------------------------------------------

def rotate_matrix(state_x):
    r_vec = state_x[0:3]
    v_vec = state_x[3:6]
    
    r_hat = r_vec/(norm(r_vec))
    

    h_vec = np.cross(r_vec, v_vec)
    h_hat = h_vec / np.linalg.norm(h_vec)  # N (Normal) direction
    t_hat = np.cross(h_hat, r_hat)  

    rotation = np.column_stack((r_hat, t_hat, h_hat))
    
    return rotation

#2.1.3 ----------------------------------------------------------------------

delta_v = 0.01
rotation_transform = rotate_matrix(X0)


def impulse(r_matrix, direct, initial_state):
    
    dv_ = np.dot(delta_v, direct)
    
    impulse_eci =  r_matrix @ dv_
    
    state_final = initial_state.copy()
    state_final[3:] += impulse_eci
    oElements_initial = RV2COE(initial_state, mu_earth)
    oElements_final = RV2COE(state_final, mu_earth)
    
    coe_diff = oElements_final - oElements_initial
    
    return coe_diff, oElements_final
    
    

# '''
# Reporting the different states for the radial, transverse and normal impulses at that point
# '''

er = [1,0,0]
et = [0,1,0]
en = [0,0,1]
# print(impulse(rotate_matrix(X0), er, X0)[0])
# print(impulse(rotate_matrix(X0), et, X0)[0])
# print(impulse(rotate_matrix(X0), en, X0)[0])


#2.1.4 ----------------------------------------------------------------------

oElements_initial = RV2COE(X0, mu_earth)
theta_array = np.arange(0,2*np.pi, 0.01)

delta_elements_radial = []
delta_elements_transverse = []
delta_elements_normal = []

all_elements_N = np.zeros((6,len(theta_array)))
i = 0
for theta in theta_array:
    
    coe = oElements_initial.copy()
    coe[5] = theta
    state_RV = np.concatenate(COE2RV(coe, mu_earth))
    
    rotation = rotate_matrix(state_RV)
    
    
    delta_elements_radial.append(impulse(rotation, er, state_RV)[0])
    delta_elements_transverse.append(impulse(rotation, et, state_RV)[0])
    delta_elements_normal.append(impulse(rotation, en, state_RV)[0])
    
    all_elements_N[0:6, i] = impulse(rotation, en, state_RV)[1]
    i = i + 1

deltaR_array = np.array(delta_elements_radial)
deltaT_array = np.array(delta_elements_transverse)
deltaN_array = np.array(delta_elements_normal)

delta_i = deltaN_array[:, 2]

absdel_i = np.abs(delta_i)
peak_widths = 75

peak_indices = signal.find_peaks_cwt(absdel_i, peak_widths)

max_delta_i = [delta_i[peak_indices[0]], delta_i[peak_indices[1]]]
theta_imax = [theta_array[peak_indices[0]], theta_array[peak_indices[1]]]
w_imax = [all_elements_N[4,peak_indices[0]], all_elements_N[4,peak_indices[1]]]

u_val = np.array(theta_imax) + np.array(w_imax)

# print(f"Maximum ∆i: {max_delta_i[0]:.4f} and {max_delta_i[1]:.4f} radians")
# print(f"Occurs at true anomaly θ = {theta_imax[0]:.2f} and {theta_imax[1]:.2f} radians")
# print(f"Here, ω = {w_imax[0]:.2f} and {w_imax[0]:.2f} radians")
# print(f"These represent a u value of {u_val[0]:.3f} and {u_val[1]:.3f}")
# print("Hence, maximum impact from impulse occurs at the preigee and apogee in the normal direction")
labels = ['a (km)', 'e', 'i (rad)', 'RAAN (rad)', 'ω (rad)', 'θ (rad)']


if show_plots:
    for i in range(6):
        plt.figure(figsize=(10, 4))
        plt.plot(theta_array, deltaR_array[:, i], label='Radial')
        plt.plot(theta_array, deltaT_array[:, i], label='Transverse')
        plt.plot(theta_array, deltaN_array[:, i], label='Normal')
        plt.title(f'Change in {labels[i]} due to ∆V')
        plt.xlabel('True Anomaly θ (deg)')
        plt.ylabel(f'∆{labels[i]}')
        plt.legend()
        plt.grid(True)
        plt.tight_layout()
        plt.show()
            
 #2.2.1 ----------------------------------------------------------------------
       
parking_altitude = 220
parking_radius = parking_altitude + radius_earth



radius_apogee = r_mean*np.arange(1.1, 2, 0.01)

semi_major_axis = 0.5*(parking_radius + radius_apogee)
#print(semi_major_axis)

transfer_e = (radius_apogee-parking_radius)/(radius_apogee+parking_radius)


cos_theta_A = (semi_major_axis * (1 - transfer_e**2) / r_mean - 1) / transfer_e


#cos_theta_A = np.clip(cos_theta_A, -1.0, 1.0)


theta_2 = np.arccos(cos_theta_A)


semi_latus_rectum =  semi_major_axis * (1 - transfer_e**2)

v_radial = np.sqrt(mu_earth / semi_latus_rectum) * transfer_e * np.sin(theta_2)
v_transverse = np.sqrt(mu_earth / semi_latus_rectum) * (1 + transfer_e * np.cos(theta_2))

normed_apogee = radius_apogee/r_mean

if show_plots:
    plt.plot(normed_apogee, v_radial, label="radial")
    plt.legend()
    plt.show()
    plt.figure()
    plt.plot(normed_apogee, v_transverse, label="transverse")
    plt.legend()
    plt.show()

#2.2.2 ----------------------------------------------------------------------
 

theta_eccentric = 2*np.arctan(np.tan(theta_2/2) * ((1-transfer_e)/(1+transfer_e))**(0.5))
theta_mean = theta_eccentric - transfer_e*np.sin(theta_eccentric)
time_total = time_orbit(semi_major_axis, mu_earth)

delta_t = (time_total / (2 * np.pi)) * theta_mean / days_convert
if show_plots:
    plt.plot(normed_apogee, delta_t)
    plt.axhline(3)
    
    
#2.2.3 ----------------------------------------------------------------------
 


v_moon = np.sqrt(mu_earth/r_mean)

v_ir = v_radial
v_it = v_transverse-v_moon
v_inf = np.sqrt(v_radial**2 + v_it**2)


delta_angle = 2* np.arctan(np.abs(v_it),v_ir)
#print(delta_angle)

hyperbolic_e = (np.sin(0.5*delta_angle))**-1
#print(hyperbolic_e)

r_perilune = (hyperbolic_e - 1)*mu_moon/(v_inf**2)

#print(r_perilune)
altitude_perilune = r_perilune - radius_moon

plt.plot(delta_t, altitude_perilune)

# If the radial velocity is flipped while the transverse velocity remains unchanged
# Then the total turning angle, delta, is twice the angle between v_inf and the moon
# in the radial direction




    
    
    
    
    
    
    
    


\end{lstlisting}

\end{document}
